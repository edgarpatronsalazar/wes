% This is samplepaper.tex, a sample chapter demonstrating the
% LLNCS macro package for Springer Computer Science proceedings;
% Version 2.21 of 2022/01/12
%
\documentclass[hf]{ceurart}
%
\usepackage[T1]{fontenc}
% T1 fonts will be used to generate the final print and online PDFs,
% so please use T1 fonts in your manuscript whenever possible.
% Other font encondi ngs may result in incorrect characters.
%
\usepackage{booktabs} 
\usepackage{amsmath}
\usepackage{caption}
\usepackage{subcaption}
%\DeclareMathOperator{\dis}{d}

\usepackage{booktabs}
\begin{document}

\copyrightyear{2024}
\copyrightclause{Copyright for this paper by its authors.
  Use permitted under Creative Commons License Attribution 4.0
  International (CC BY 4.0).}

\conference{Research Fundamentals - Dr. Mauricio G. Orozco del Castillo}

%
% titulo para journal
% An Explainable Sliding-Window Method for Weather Forecasting

\title{Control de movimientos en robots: Instrucciones precisas para el movimiento optimo.}

%\titlerunning{A Case-based Explanation Method for Weather Forecasting}

\author[1]{Edgar Rubén Patrón-Salazar}[email=le24080054@merida.tecnm.mx]
%\author[1]{Gerardo Arturo Pérez-Pérez}

\cormark[1]

\address[1]{Tecnológico Nacional de México/IT de Mérida,
  Department of Systems and Computing, Merida, Mexico}
%
\maketitle
%
\section{Ensayo}
La programación de movimientos básicos en robots plantea desafíos intrigantes y abre oportunidades emocionantes en el ámbito de la robótica. En este ensayo, exploraremos las suposiciones y reflexiones derivadas del ejercicio de proporcionar instrucciones para dirigir un robot, tomando en cuenta nuestra perspectiva aérea en el proceso.\\
En el ejercicio propuesto, las instrucciones para controlar movimientos se centran en aspectos cruciales como la posición y la orientación del robot. Se asume que el robot tiene la capacidad de comprender y ejecutar comandos precisos, lo cual refleja una combinación de hardware y software avanzado.\\
Una de las suposiciones clave es la importancia de mantener la consistencia en la orientación del robot durante diversos movimientos. Esto plantea la cuestión de cómo lograr giros y desplazamientos sin perder la referencia espacial.\\
La tarea también implica la necesidad de instrucciones claras y concisas, asumiendo que la comunicación entre el programador y el robot se basa en un lenguaje específico. Aquí surge la pregunta sobre cómo optimizar las instrucciones para garantizar que sean fácilmente interpretadas y ejecutadas por el robot.\\
En conclusión, el ejercicio de proporcionar instrucciones para la programación de movimientos básicos en robots nos lleva a reflexionar sobre las complejidades y las posibilidades que surgen en este campo. Las suposiciones sobre la capacidad del robot, la interpretación de instrucciones y la consistencia en la orientación plantean desafíos intrigantes. Además, sugiere beneficios tangibles en otros ámbitos, como la capacidad para aprender a dar órdenes lógicas. La habilidad de comunicar instrucciones de manera clara y coherente es crucial no solo en la programación de robots, sino también en interacciones cotidianas. Este proceso de dar instrucciones lógicas podría aplicarse, por ejemplo, en la comunicación con sistemas como ChatGPT, permitiendo una interacción más efectiva y comprensible.


\end{document}
